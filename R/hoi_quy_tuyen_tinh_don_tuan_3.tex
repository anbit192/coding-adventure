% Options for packages loaded elsewhere
\PassOptionsToPackage{unicode}{hyperref}
\PassOptionsToPackage{hyphens}{url}
%
\documentclass[
]{article}
\usepackage{amsmath,amssymb}
\usepackage{lmodern}
\usepackage{iftex}
\ifPDFTeX
  \usepackage[T1]{fontenc}
  \usepackage[utf8]{inputenc}
  \usepackage{textcomp} % provide euro and other symbols
\else % if luatex or xetex
  \usepackage{unicode-math}
  \defaultfontfeatures{Scale=MatchLowercase}
  \defaultfontfeatures[\rmfamily]{Ligatures=TeX,Scale=1}
\fi
% Use upquote if available, for straight quotes in verbatim environments
\IfFileExists{upquote.sty}{\usepackage{upquote}}{}
\IfFileExists{microtype.sty}{% use microtype if available
  \usepackage[]{microtype}
  \UseMicrotypeSet[protrusion]{basicmath} % disable protrusion for tt fonts
}{}
\makeatletter
\@ifundefined{KOMAClassName}{% if non-KOMA class
  \IfFileExists{parskip.sty}{%
    \usepackage{parskip}
  }{% else
    \setlength{\parindent}{0pt}
    \setlength{\parskip}{6pt plus 2pt minus 1pt}}
}{% if KOMA class
  \KOMAoptions{parskip=half}}
\makeatother
\usepackage{xcolor}
\usepackage[margin=1in]{geometry}
\usepackage{color}
\usepackage{fancyvrb}
\newcommand{\VerbBar}{|}
\newcommand{\VERB}{\Verb[commandchars=\\\{\}]}
\DefineVerbatimEnvironment{Highlighting}{Verbatim}{commandchars=\\\{\}}
% Add ',fontsize=\small' for more characters per line
\usepackage{framed}
\definecolor{shadecolor}{RGB}{248,248,248}
\newenvironment{Shaded}{\begin{snugshade}}{\end{snugshade}}
\newcommand{\AlertTok}[1]{\textcolor[rgb]{0.94,0.16,0.16}{#1}}
\newcommand{\AnnotationTok}[1]{\textcolor[rgb]{0.56,0.35,0.01}{\textbf{\textit{#1}}}}
\newcommand{\AttributeTok}[1]{\textcolor[rgb]{0.77,0.63,0.00}{#1}}
\newcommand{\BaseNTok}[1]{\textcolor[rgb]{0.00,0.00,0.81}{#1}}
\newcommand{\BuiltInTok}[1]{#1}
\newcommand{\CharTok}[1]{\textcolor[rgb]{0.31,0.60,0.02}{#1}}
\newcommand{\CommentTok}[1]{\textcolor[rgb]{0.56,0.35,0.01}{\textit{#1}}}
\newcommand{\CommentVarTok}[1]{\textcolor[rgb]{0.56,0.35,0.01}{\textbf{\textit{#1}}}}
\newcommand{\ConstantTok}[1]{\textcolor[rgb]{0.00,0.00,0.00}{#1}}
\newcommand{\ControlFlowTok}[1]{\textcolor[rgb]{0.13,0.29,0.53}{\textbf{#1}}}
\newcommand{\DataTypeTok}[1]{\textcolor[rgb]{0.13,0.29,0.53}{#1}}
\newcommand{\DecValTok}[1]{\textcolor[rgb]{0.00,0.00,0.81}{#1}}
\newcommand{\DocumentationTok}[1]{\textcolor[rgb]{0.56,0.35,0.01}{\textbf{\textit{#1}}}}
\newcommand{\ErrorTok}[1]{\textcolor[rgb]{0.64,0.00,0.00}{\textbf{#1}}}
\newcommand{\ExtensionTok}[1]{#1}
\newcommand{\FloatTok}[1]{\textcolor[rgb]{0.00,0.00,0.81}{#1}}
\newcommand{\FunctionTok}[1]{\textcolor[rgb]{0.00,0.00,0.00}{#1}}
\newcommand{\ImportTok}[1]{#1}
\newcommand{\InformationTok}[1]{\textcolor[rgb]{0.56,0.35,0.01}{\textbf{\textit{#1}}}}
\newcommand{\KeywordTok}[1]{\textcolor[rgb]{0.13,0.29,0.53}{\textbf{#1}}}
\newcommand{\NormalTok}[1]{#1}
\newcommand{\OperatorTok}[1]{\textcolor[rgb]{0.81,0.36,0.00}{\textbf{#1}}}
\newcommand{\OtherTok}[1]{\textcolor[rgb]{0.56,0.35,0.01}{#1}}
\newcommand{\PreprocessorTok}[1]{\textcolor[rgb]{0.56,0.35,0.01}{\textit{#1}}}
\newcommand{\RegionMarkerTok}[1]{#1}
\newcommand{\SpecialCharTok}[1]{\textcolor[rgb]{0.00,0.00,0.00}{#1}}
\newcommand{\SpecialStringTok}[1]{\textcolor[rgb]{0.31,0.60,0.02}{#1}}
\newcommand{\StringTok}[1]{\textcolor[rgb]{0.31,0.60,0.02}{#1}}
\newcommand{\VariableTok}[1]{\textcolor[rgb]{0.00,0.00,0.00}{#1}}
\newcommand{\VerbatimStringTok}[1]{\textcolor[rgb]{0.31,0.60,0.02}{#1}}
\newcommand{\WarningTok}[1]{\textcolor[rgb]{0.56,0.35,0.01}{\textbf{\textit{#1}}}}
\usepackage{graphicx}
\makeatletter
\def\maxwidth{\ifdim\Gin@nat@width>\linewidth\linewidth\else\Gin@nat@width\fi}
\def\maxheight{\ifdim\Gin@nat@height>\textheight\textheight\else\Gin@nat@height\fi}
\makeatother
% Scale images if necessary, so that they will not overflow the page
% margins by default, and it is still possible to overwrite the defaults
% using explicit options in \includegraphics[width, height, ...]{}
\setkeys{Gin}{width=\maxwidth,height=\maxheight,keepaspectratio}
% Set default figure placement to htbp
\makeatletter
\def\fps@figure{htbp}
\makeatother
\setlength{\emergencystretch}{3em} % prevent overfull lines
\providecommand{\tightlist}{%
  \setlength{\itemsep}{0pt}\setlength{\parskip}{0pt}}
\setcounter{secnumdepth}{-\maxdimen} % remove section numbering
\ifLuaTeX
  \usepackage{selnolig}  % disable illegal ligatures
\fi
\IfFileExists{bookmark.sty}{\usepackage{bookmark}}{\usepackage{hyperref}}
\IfFileExists{xurl.sty}{\usepackage{xurl}}{} % add URL line breaks if available
\urlstyle{same} % disable monospaced font for URLs
\hypersetup{
  pdftitle={HQTT don tuan 3},
  hidelinks,
  pdfcreator={LaTeX via pandoc}}

\title{HQTT don tuan 3}
\author{}
\date{\vspace{-2.5em}2023-03-02}

\begin{document}
\maketitle

\hypertarget{r-markdown}{%
\subsection{R Markdown}\label{r-markdown}}

\begin{Shaded}
\begin{Highlighting}[]
\FunctionTok{data}\NormalTok{(cars)}
\end{Highlighting}
\end{Shaded}

\begin{enumerate}
\def\labelenumi{\arabic{enumi}.}
\tightlist
\item
  Chia bộ dữ liệu cars thành hai bộ dữ liệu con: training (65\%) và
  testing (35\%)
\end{enumerate}

\begin{Shaded}
\begin{Highlighting}[]
\CommentTok{\# sample() tạo ra bảng các giá trị dưa vào dữ liệu được đưa vào.}
\CommentTok{\# sample(x, size, replace = FALSE, prob = NULL)}


\CommentTok{\#Áp dụng để chia bộ dữ liệu thành 2 phần}

\CommentTok{\#Tạo ra bảng các giá trị ngẫu nhiên true hoặc false}

\NormalTok{size }\OtherTok{=} \FunctionTok{nrow}\NormalTok{(cars) }

\NormalTok{sample }\OtherTok{=} \FunctionTok{sample}\NormalTok{(}\FunctionTok{c}\NormalTok{(}\ConstantTok{TRUE}\NormalTok{, }\ConstantTok{FALSE}\NormalTok{), size, }\AttributeTok{replace =} \ConstantTok{TRUE}\NormalTok{, }\AttributeTok{prob =} \FunctionTok{c}\NormalTok{(}\FloatTok{0.65}\NormalTok{, }\FloatTok{0.35}\NormalTok{))}
\NormalTok{sample}
\end{Highlighting}
\end{Shaded}

\begin{verbatim}
##  [1]  TRUE  TRUE  TRUE  TRUE  TRUE  TRUE  TRUE  TRUE FALSE  TRUE FALSE  TRUE
## [13]  TRUE  TRUE FALSE  TRUE  TRUE  TRUE  TRUE  TRUE FALSE  TRUE  TRUE  TRUE
## [25] FALSE FALSE  TRUE  TRUE FALSE  TRUE FALSE  TRUE  TRUE  TRUE  TRUE  TRUE
## [37] FALSE  TRUE  TRUE  TRUE FALSE  TRUE FALSE  TRUE FALSE FALSE FALSE FALSE
## [49]  TRUE  TRUE
\end{verbatim}

\begin{Shaded}
\begin{Highlighting}[]
\CommentTok{\# vector(c(TRUE, FALSE)): các giá trị trong bảng (có thể là bất kì giá trị nào)}
\CommentTok{\# size : \# số lượng phần tử trong bảng.}
\CommentTok{\# replace: }
\CommentTok{\# prob: khoảng phân bố giữa các giá trị}
\end{Highlighting}
\end{Shaded}

=\textgreater{} Bảng sample lúc này sẽ bao gồm 50 (vì nrow(cars) = 50)
giá trị TRUE/FALSE

=\textgreater{} Mỗi giá trị TRUE/FALSE sẽ tương ứng với một hàng trong
bộ dữ liệu cars

Ví dụ:

\begin{verbatim}
speed dist 
\end{verbatim}

1 4 2 TRUE 2 4 10 FALSE \ldots{} \ldots{} \ldots{} \ldots{}

\begin{Shaded}
\begin{Highlighting}[]
\FunctionTok{table}\NormalTok{(sample)}
\end{Highlighting}
\end{Shaded}

\begin{verbatim}
## sample
## FALSE  TRUE 
##    15    35
\end{verbatim}

\begin{Shaded}
\begin{Highlighting}[]
\FunctionTok{set.seed}\NormalTok{(}\DecValTok{1}\NormalTok{)}

\NormalTok{training }\OtherTok{=}\NormalTok{ cars[sample, ]}
\NormalTok{testing }\OtherTok{=}\NormalTok{ cars[}\SpecialCharTok{!}\NormalTok{sample, ]}
\NormalTok{testing}
\end{Highlighting}
\end{Shaded}

\begin{verbatim}
##    speed dist
## 9     10   34
## 11    11   28
## 15    12   28
## 21    14   36
## 25    15   26
## 26    15   54
## 29    17   32
## 31    17   50
## 37    19   46
## 41    20   52
## 43    20   64
## 45    23   54
## 46    24   70
## 47    24   92
## 48    24   93
\end{verbatim}

\begin{Shaded}
\begin{Highlighting}[]
\FunctionTok{dim}\NormalTok{(training)}
\end{Highlighting}
\end{Shaded}

\begin{verbatim}
## [1] 35  2
\end{verbatim}

\begin{Shaded}
\begin{Highlighting}[]
\FunctionTok{dim}\NormalTok{(testing)}
\end{Highlighting}
\end{Shaded}

\begin{verbatim}
## [1] 15  2
\end{verbatim}

\begin{itemize}
\tightlist
\item
  training data: chỉ lấy những hàng trong bộ cars có chỉ số index là
  sample = TRUE. Ví dụ: speed dist 1 4 2 TRUE 2 4 10 FALSE \ldots{}
  \ldots{} \ldots{} \ldots{}
\end{itemize}

=\textgreater{} hàng 1 của bộ cars là TRUE

+testing data: tương tự với sample = FALSE

\begin{enumerate}
\def\labelenumi{\arabic{enumi}.}
\setcounter{enumi}{1}
\tightlist
\item
  Phân tích hồi quy tuyến tính của biến dist theo speed trên bộ dữ liệu
  training.
\end{enumerate}

\begin{Shaded}
\begin{Highlighting}[]
\NormalTok{lmfit }\OtherTok{=} \FunctionTok{lm}\NormalTok{(training}\SpecialCharTok{$}\NormalTok{dist }\SpecialCharTok{\textasciitilde{}}\NormalTok{ training}\SpecialCharTok{$}\NormalTok{speed)}
\NormalTok{lmfit}
\end{Highlighting}
\end{Shaded}

\begin{verbatim}
## 
## Call:
## lm(formula = training$dist ~ training$speed)
## 
## Coefficients:
##    (Intercept)  training$speed  
##        -18.772           4.054
\end{verbatim}

\begin{Shaded}
\begin{Highlighting}[]
\FunctionTok{summary}\NormalTok{(lmfit)}
\end{Highlighting}
\end{Shaded}

\begin{verbatim}
## 
## Call:
## lm(formula = training$dist ~ training$speed)
## 
## Residuals:
##     Min      1Q  Median      3Q     Max 
## -30.298 -10.004  -4.405   7.157  42.023 
## 
## Coefficients:
##                Estimate Std. Error t value Pr(>|t|)    
## (Intercept)     -18.772      8.445  -2.223   0.0332 *  
## training$speed    4.053      0.551   7.357  1.9e-08 ***
## ---
## Signif. codes:  0 '***' 0.001 '**' 0.01 '*' 0.05 '.' 0.1 ' ' 1
## 
## Residual standard error: 16.84 on 33 degrees of freedom
## Multiple R-squared:  0.6212, Adjusted R-squared:  0.6097 
## F-statistic: 54.12 on 1 and 33 DF,  p-value: 1.901e-08
\end{verbatim}

y = -16.609 + 3.843*x

\begin{enumerate}
\def\labelenumi{\arabic{enumi}.}
\setcounter{enumi}{2}
\tightlist
\item
  Đánh giá trên bộ dữ liệu testing.
\end{enumerate}

R\^{}2 = 1 - (RSS/TSS)

TSS (Total sum of square) = sum( {[}y - mean(y){]}\^{}2 ) RSS = sum(
{[}y - a0h - a1h * x{]}\^{}2 )

\begin{Shaded}
\begin{Highlighting}[]
\NormalTok{y\_h }\OtherTok{=}\NormalTok{ lmfit}\SpecialCharTok{$}\NormalTok{coefficients[}\DecValTok{1}\NormalTok{] }\SpecialCharTok{+}\NormalTok{ lmfit}\SpecialCharTok{$}\NormalTok{coefficients[}\DecValTok{2}\NormalTok{] }\SpecialCharTok{*}\NormalTok{ testing}\SpecialCharTok{$}\NormalTok{speed}
\NormalTok{tss }\OtherTok{=} \FunctionTok{sum}\NormalTok{((testing}\SpecialCharTok{$}\NormalTok{dist }\SpecialCharTok{{-}} \FunctionTok{mean}\NormalTok{(testing}\SpecialCharTok{$}\NormalTok{dist))}\SpecialCharTok{\^{}}\DecValTok{2}\NormalTok{)}
\NormalTok{tss}
\end{Highlighting}
\end{Shaded}

\begin{verbatim}
## [1] 6575.6
\end{verbatim}

\begin{Shaded}
\begin{Highlighting}[]
\NormalTok{rss }\OtherTok{=} \FunctionTok{sum}\NormalTok{((testing}\SpecialCharTok{$}\NormalTok{dist }\SpecialCharTok{{-}}\NormalTok{ y\_h)}\SpecialCharTok{\^{}}\DecValTok{2}\NormalTok{)}
\NormalTok{rss}
\end{Highlighting}
\end{Shaded}

\begin{verbatim}
## [1] 2032.862
\end{verbatim}

\begin{Shaded}
\begin{Highlighting}[]
\NormalTok{r2 }\OtherTok{=} \DecValTok{1} \SpecialCharTok{{-}}\NormalTok{ rss}\SpecialCharTok{/}\NormalTok{tss}
\NormalTok{r2}
\end{Highlighting}
\end{Shaded}

\begin{verbatim}
## [1] 0.6908477
\end{verbatim}

=\textgreater{} Testing: R2 = 0.5593682 Training: R2 = 0.7237
=\textgreater{} Bộ data testing là ko tốt vì giá trị R2 bị giảm so với
bộ training

\end{document}
